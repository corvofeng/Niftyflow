% vim: foldmethod=marker foldcolumn=3

% {{{ 绪论
\chapter{绪论}
\section{选题依据及意义}

  数据中心(DCN)在目前的企业中被广泛采用, 更是云计算环境的运行基础,
它具有规模大、可靠性要求高、应用多样等特点.
用户对网络性能的要求日益扩大, 硬件故障、配置失误等均会造成
网络服务质量和用户的使用体验下降.
特别是在某些交互敏感的场合, 服务的质量必须得到保证.
但是, 采用小而非零故障率的硬件意味着数据中心会遭遇大量故障.
我们只能更快、更精准的解决问题, 将故障的损失降到最低, 这也是对新时代的
检测调试工具的挑战. 及时发现故障并确定故障位置成为DCN网络运维的重要组成部分.

  理解和调试DCN中的错误是很有挑战的, 因为错误多种多样, 例如:
\begin{itemize}
  \setlength\itemsep{-0.3em}
  \item 服务器之间某些数据包的延迟过大, 我们无法知道是否是链路存在问题;
  \item 通往特定服务器的数据包丢失, 而交换机上的丢包计数器也没有表现异常;
  \item 链路负载不均衡, 而管理员无法判断是流量大小差异还是更深的问题.
\end{itemize}

  这些问题的诊断, 需要我们在数据包的粒度上进行测试. 类似于交换机的软件故障以及
有错误的接口, 很可能导致随机的错误, 并且影响具有特定特征的数据包, 这些特征可能
是路由选择、数据包头部、或是达到时间. 由于这些微妙的影响, 上述错误
往往难以通过流级别的信息, 汇总计数器或是采样流量的分析进行调试. 我们需要
追踪、收集、分析数据包级别的流量行为来进行网络检测.

建立可靠的包级别的探测系统有三个挑战:首先, 当前的DCN具有史无前例的网络流量,
一个庞大的DCN拥有超过10万个服务器, 每个都有10-40Gbps的网络流量. 在高利用率下,
总流量很容易超过100Tbps. 对于今天的商用交换机来说, 分析其中的一部分流量都是棘手的,
并且将流量移动到服务器中分析会导致网络拥塞, 甚至破坏网络.

其次, DCN的错误经常会发生在多跳或是多个交换机中, 并且有效的追踪需要智能的
追溯网络中的一小部分包, 需要基于复杂的匹配模式去查找, 例如协议头、起始点、
甚至是路径中的设备, 追踪难度巨大. 而目前存在的基于包级别的分析是追踪所有数据包,
对于大规模的DCN或是复杂的匹配模式, 它们也不能提供扩展性.

而后, 被动的追踪瞬间的网络快照, 将会被性能所限制. 快照中的信息可能不足以判断
问题是暂时的或是永久的. 也不能提供足够的信息去定位错误.
例如, 我们看到一组包在到达终点时停止,
我们无法判断出这是由于随机丢弃或是黑洞.

Everflow是微软在2015年发表在SIGCOMM上的论文提出的方案. 该方案通过使用
商用交换机的"匹配镜像"机制, 实现了一套的数据包过滤器.
镜像特定数据包并还原其路径, 以此分析数据中心网络中的故障,
也支持发送探针报文来确认网络中的潜在问题.

\section{研究目标和主要内容}

  图\ref{fig:3level}所示的网络架构(三层网络结构, 分别为
接入层、汇聚层、核心层)为本次设计面向的DCN网络结构.
对于其中可能存在的类似丢包、环路等问题. 我们需要在数据包层面进行分析,
向网络管理员提供问题发生的确切原因.

  本课题参照微软提出的Everflow检测方案, 基于通用处理器平台(x86-64)进行构建.
设计实现了一套完整的流量分析及预警展示工具.
具有良好的可扩展性, 为专有平台的程序编写提供参照. 本次的任务包括:

\begin{itemize}
  \item 对实时采集到的数据包进行分析, 实现丢包检测、环路分析等故障检测;
  \item 按需保存Trace数据并在必要时生成主动探测包, 以进一步确定故障位置和故障原因;
  \item 针对上述故障检测需求进行应用界面的开发. 具备统计分析、异常报警等功能.
\end{itemize}

\begin{figure}[htbp!]
  \centering
  \includegraphics[width=0.8\textwidth]{../img/3level.png}
  \caption{三层网络架构}
  \label{fig:3level}
\end{figure}


下面列出了程序的主要功能:
\begin{itemize}
  \item
    对实时采集到的数据包进行分析, 实现丢包检测、环路分析等故障检测;
  \item
    采用两阶段的分析方案, 分别为快路径与慢路径;
  \item
    在分析时, 多处使用hash数据结构来提高查找效率, 加快处理速度;
  \item
    按需保存Trace数据, 以进一步确定故障位置和故障原因;
  \item
    支持探针数据报文的发送(需要管理员手动触发);
  \item
    采用B/S架构, 管理员通过浏览器即可进行操作。
\end{itemize}
% }}}

% {{{ 国内外研究现状
\chapter{国内外研究现状}
\section{数据中心网络故障检测方案}

不论是学校还是企业, 许多都设立了自己的数据中心,
有关数据中心的网络故障检测, 都有自己的见解与考量. 对于更为大型的数据中心网络,
更是有人提出了集学校公司之力, 共同开发系统, 不断的完善.

经过对国内外近期文献的查阅, 我发现,
这些系统或是工具可以按照不同角度进行分类. 在主动性方面,
有主动探测故障的, 有被动触发, 还有使用日志记录来查看的; 在网络层次方面,
有些在数据包层面, 有些则专注于数据流层面; 根据部署位置不同,
也可以分为在交换机部署, 在终端部署, 或者两者均进行部署; 在使用范围上,
有些适合于通用平台, 有些只能适用于某个特定厂商; 在处理上,
有些拥有流水线来处理结果, 有些则是简单的排除错误; 按工具性质来看,
有些属于小的调试工具, 专注于某个方面, 也不需要一直在线,
有些则是属于系统级别的服务工具, 为整个DCN所使用, 需要一直在线;
在错误处理上, 有些可以借助控制设备自动进行处理,
而有些则需要向管理员预警, 需管理员手动进行处理. 以下介绍各个方案的一些特点.

Planck:\cite{rasley2014planck} SDN的出现使得自己调节的网络得以实现, 这样
的网络可以实时监控, 并能立即对重要的事件例如拥塞做出迅速反应.
但是目前的监控机制需要数百毫秒来重新探测全局网络, 对于实时的错误,
这样的延迟是很致命的, 这篇2014年的论文提出了新颖的网络测量架构,
使用端口镜像的机制来提取网络信息. 但是可以在相当
短的时间内对网络信息进行获取. 而且不会对网络造成太大的影响.

LossRadar:\cite{li2016lossradar} 属于有针对性的工具,
着重介绍有关丢包的抓取问题. 虽然只是检测丢包,
但是这也足够成为一个检测系统了, 这个工具可以在较快的时间
内抓取单独的丢包以及它们的详细信息. 它需要在交换机上部署,
但是并不需要很多的流量和带宽.

Cherrypick:\cite{tammana2015cherrypick} 一个可扩展的简单的轨迹追踪技术.
目前的数据轨迹追踪需要负担大量数据积累的开销,
或是在数据平面大量资源的消耗, 交换机规则或是数据包头部的探查.
它的核心思想是挑选链接, 这些链接是表示数据包端到端路径的关键,
并在到达目的地的路上将其嵌入数据包头部. 通过使用最新的头部标识技术,
它只需要很少的交换机规则即可,

SDN traceroute:\cite{agarwal2014sdn} 使用具有了SDN功能的设备, 不过只要
支持OpenFlow1.0的设备即可. 它可以通过SDN支持的网络中的任意数据包来确定路径,
这个路径使用SDN支持的转发机制, 而且并不需要修改转发规则.
可以探测任意的以太网数据包的转发行为, 以及交换机和控制器逻辑中的调试问题.

PathDump:\cite{tammana2016simplifying}使用了较为不同的方法,
仔细划分了边缘设备与网络元素之间的调试任务.
利用边缘设备的资源进行网络调试的简约工具. 并且可以支持大量网络调试问题.
需要的资源较少, 而且可以在较细的时间粒度调试.

Pingmesh:\cite{guo2015pingmesh}
这个系统已经在微软的数据中心部署了超过4年, 它的理念很简单,
就是想要在任意时刻获取任意两台服务器的延迟信息. 因此, 它的目标就是
去定义一个网络延迟检测和分析系统, 它也需要为所有的服务器产生延迟信息,
因为延迟数据属于基础信息, 能够帮助我们更好的管理网络,
以及解决网络中的问题. 这项服务也必须长期在线, 并保证稳定性.
它可能是最不需要关心路由器等设备的一个方案了, 对于整个系统来说,
知道延迟就是目标, 在实现时, 主要借助了ping的思想,
所有服务器要从中心控制器下载文件, 进行延迟探测后传回中心服务器.
它借助了微软自己开发了存储系统, 也实现了 数据处理流水线.
相比于之后提出的Everflow, 它更像是微软的第一代产品, 稳定而强大.

NetSight:\cite{handigol2014know}, 这是一个完全记录网络历史的工具.
在文章中, 介绍了如何使用packet histories(每个包的整条记录)来简化网络的调试.
为了展示\texttt{packet\ histories}的作用, 以及实现上的可行性,
创造了\texttt{NetSight}, 一个可扩展的平台,
允许程序简便的检索网络的历史状况. 在\texttt{NetSight}上, 有4个程序:
可交互的网络调试器, 实时的监视器, 一个历史记录器, 一个分级分析器.
在一个现代的多核服务器上,
\texttt{NetSight}可以在10G/s的链接中处理历史包. 对于更大的网络,
它可以通过增加服务器或是硬件或是交换机的数量来扩展. 需要借助SDN,
支持Openflow的交换机才能配合工作, 由于其记录了数据包的历史, 所以
能最大程度还原当初的网络模型, 也因此会给网络带来高负荷, 性能也会损失,
当然也可以通过增加硬件来进行提升. 这是一个与Everflow较为相似的产品,
Everflow中对其一个重要的改进就是增加了匹配模式, 对特定的历史包生成记录,
能有效减少网络负载, 以及处理难度.

Netography:\cite{zhao2016netography} 这也是一种基于软件定义网络(SDN)的
一个工具, 之前的工作关注于静态检查、被动监控、以及活动探针,
这些依赖于控制设备以及网络设备的抓取规则. 它定义了一个数据包行为的概念,
用以描述数据包的真实变化, 并强调对故障排除的重要性. 基于通过
由主动发送的探测器触发的副本导出分组行为和流规则的新颖方法,
提出了Netography系统, 并说明了关于转发错误时的排除任务过程,
以及由非租户争用造成的性能下降问题.

Dissecting RTT(round trip time):\cite{marchetta2014dissecting} 与其说它
是工具, 不如认为这一种技术, 利用单个数据包探测延迟的技术. 研究人员
与操作者经常会在监控、故障排除、或是其他方式访问网络路径时测量往返时间.
因为延迟时间中结合了所有跳的过程以及转发和反向路径,
所以很难去衡量特定网络元素的延迟.
在这项工作中, 提出了一种新的方法:在区块中, 映射特定路径后
基于单个数据包探测往返延迟. 使用针对中间路由器的IP Prespecified
Timestamp选项, 它可以提供慢速路径部分的往返时间估计.
这个技术在Everflow中也被用到了, 用于探针方法测量时延.

Passive Realtime Datacenter:\cite{roy2017passive} 这是在Facebook的数据
中心上进行测试的一种方案, 它不去简单的观察异常现象,
而是考虑异常对整个系统的性能所造成的影响. 虽然是基于这么一种简单的设想.
但其实现需要深深结合数据中心的架构. 他们开发了轻量级的包标记技术,
仅仅使用转发规则(交换机支持). 作为路径中唯一标示. 文章中对于现有IPv6头部, 
还很难寻求一份区域用于调试.


\section{Everflow简介}

  Everflow \cite{greenberg2016packet} 这是微软在2015年发表在
SIGCOMM的论文中提出的方案. 与部署运行超过了4年的Pingmesh\cite{guo2015pingmesh}
不同, 它并没有特别关注端对端的延迟, 而是从数据包层面进行分析.

  对于Everflow来说, 最大的挑战莫过于扩展性了. 大型DCN(10W以上的服务器)中流量
轻松就能达到100Tbps, 追踪这样规模的数据包需要大量的网络资源与服务器资源,
想象一种典型的情况, 平均每个包的大小为1000bytes,
每个镜像数据包64bytes(只包括头部信息), 并且网络的直径为5跳. 如果我们
简单的追踪每个数据包的每一跳作为数据包历史, 数据流量就会有这么多
$\frac{64B}{1000B} \times 5(hops) \times 1000(Tbps) = 32(Tbps)$, 如此高的追踪流量
将会造成网络拥塞和丢包, 特别是当网络利用率很高时.

  另一个扩展性的挑战是路径分析. 因为商用交换机有限的存储与CPU运算能力, 追踪数据包
必须由服务器来完成. 即使假设一个服务可以处理10Gbps的流量, 我们需要
$\frac{32Tbps}{10Gbps} = 3200$这么多服务器来处理, 这也是很大的开销.


\begin{figure}
  \centering
  \includegraphics[width=0.58\textwidth]{../img/everflow_arch.png}
  \caption{Everflow的路径收集与分析流水线}
  \label{fig:everflow_arch}
\end{figure}

  如图\ref{fig:everflow_arch}所示为Everflow的架构.它由四个主要组件组成:
控制器、分析器、存储器以及洗牌器. 在顶层, 有许多应用程序使用由Everflow提供的信息
去调试网络错误. 控制器负责协调其它组件以及和应用程序交流. 在
初始化过程中, 它配置交换机上的规则. 被匹配到的数据包将被镜像至洗牌器, 而后会直接
到达分析器中, 分析完成后, 分析结果将传递至存储器中.
控制器也提供了一些API, 允许Everflow的应用程序去查询分析结果, 并且在主机上设置debug位.

  Everflow的关键思想在于镜像特定的报文, 在商用交换机上设置特定的匹配规则, 将匹配后的数据包发出.
而后利用交换机内置的ASIC芯片的负载均衡, 捕获到的数据包
会被重新分发到不同的分析服务器上, 通过五元组信息(源IP、目的IP、源端口、目的端口、协议类型)
能保证在同一轨迹上的数据包能分发至相同的服务器上,
它也能发送定向探针去确认潜在的错误.

  2014年, 微软工作人员在微软两个的数据中心上进行Everflow部署.
第一个部署的是前期制作的集群, 其中有37台交换机. 第二个部署是对生产环境的
2500台交换机中的440台进行部署. 两个集群承载了许多应用程序的流量.
除此之外, 工作人员也在特定的生产交换机上开启了Everflow, 用以支持在线错误的调试.
目前, 微软工作人员也扩展了Everflow的部署以使其可以被更多的交换机和集群使用.
通过部署在微软的DCN上, Everflow证明了其作为诊断DCN错误不可或缺的工具的实用性.
% }}} Everflow简介

% {{{ 流量分析与故障检测总体方案
\chapter{流量分析与故障检测总体方案}

本文实现的流量分析与故障检测方案包括三大部分:分析器、控制器、以及访问页面, 以下分别进行概述,
更加详细的实现细节请查看接下来三章的设计方案.

\begin{figure}
  \centering
  \includegraphics[width=0.55\textwidth]{../img/user_flow.png}
  \caption{架构设计}
  \label{fig:arch}
\end{figure}

图\ref{fig:arch}为整个系统的结构. 镜像后的数据包传递至分析器中, 分析器检查过后,
异常的数据以及统计结果将会保存在存储器中, 用户可以通过访问页面来查询问题数据.

\section{分析器功能设计}

  分析器是分布式的服务器, 且每个服务器处理一部分流量数据.

  对于特定模式的数据包进行统计, 通过统计数据, 可以获得当前网络中的流量状况.

  在分析器中, 我们期望获得的是在DCN中数据包的完整路径. 对于存在问题的路径信息,
我们可以得知是否有环、或是有丢包现象, 存在以上问题的路径信息将会存储至存储器中.

  对于探针来说也可以粗略衡量交换机之间的延迟.

\section{存储器功能设计}

存储器中, 主要存储分析器产生的结果, 包括网络流数据与相应规则的统计数据.
本方案中, 使用关系型数据库MySQL\cite{mysql}进行数据存储.

网络流数据表格中, 每一行数据对应给一个追踪记录,
包括数据包的原始信息(第一个数据包的达到时间、五元组信息)、
数据包的多跳信息、当前网络流的元数据信息(是否有环、是否丢包、是否为探针).
对其中一些信息, 可以在查询时可以指定规则来进行过滤查找.

我们也在表格中存储了所有分析器产生的计数值, 表中的每一行代表着从分析器取得的
一条计数信息, 包括计数器的ID、记录时间、分析器ID.

同时, 存储器中, 也存储了全部的统计规则.

\section{控制器功能设计}

控制器有几个主要功能:对分析器配置、向外提供API接口、向交换机中发送探针.
目前我们提供的API包括:按时间查找数据流、查询计数器的值、添加探针,
发生错误后的预警功能.

\section{访问页面功能设计}

此部分用于展示存储器中的数据, 包括网络流数据, 不同的规则下的统计数据.
另外提供用户与程序的交互接口, 包括增加删除统计规则, 探针的构造及发送.

\section{技术选型}

根据对性能的不同要求, 结合开发进度做如下讨论. 在分析器中,
需要考虑大量的流量涌来, 必须充分利用系统资源进行数据包的接受和处理,
使用偏向底层的C/C++语言, 在控制器中, 只涉及简单的数据库查找与用户交互,
对性能要求不高, 故使用Python, 以实现迅速而敏捷的开发.

对于本次的系统设计, 分工如下:

\begin{itemize}
\item
  分析器采用C/C++完成;
\item
  选择关系型数据库MySQL作为存储器;
\item
  控制器通过Python实现, 与用户的接口采用HTTP协议交互;
\item
  访问页面采用HTML+CSS+Javascript实现, 图表展示打算使用开源的echarts。
\end{itemize}
% }}}流量分析与故障检测总体方案

% {{{ 实现方案
% {{{ 分析器实现方案
\chapter{分析器的设计与实现}

\section{算法设计}
\label{sec:分析器主要算法说明}

\textbf{环路检测算法}:环路检测是设计中最为基本的功能, 目前的程序中只是做简单的环路检测:
 遍历数据路径的所有节点(交换机ID), 如果存在相同的节点, 那么认为此路径中有环.

\textbf{检测丢包算法}:查看路径的最后一跳是否与我们期望的最后一跳相同.
最后一跳的交换机是无法直接计算得到的, 需要根据数据中心的网络拓扑进行分析.
当前程序中, 控制器下发出口交换机的ID信息,
任何数据包路径中均需要包含两个出口交换机, 我们才能认为它没有在网络中被丢弃.


\textbf{使用探针检测任意交换机的延迟}:这是一个比较具有考量的方法.
需要交换机具有解封和转发能力.

首先介绍如何实现一跳的探针:假设我们想要将数据包p发往S,
我们首先根据p构造出\(p^{'}\) 其中\(p^{'}\)的目的IP为S,
而后将\(p^{'}\)发送出去, 当它到达交换机S后, 数据包中的
目的IP与S的IP相同, 则将数据包\(p^{'}\)进行解封操作, 将其还原为p,
再使用正常的转发规则处理p.

实现了一跳的探针后, 可以考虑将探针进行拓展,
使其按找我们理想的路径进行传递, 方式也不难理解,
就是一层层的进行数据包封装, 这样数据包到达交换机被解封后, 就可以
根据内层的数据包进行正常转发了, 从而达到内层数据包所要去的地址中.
对于端到端的延迟, 例如我们想要知道\(S_{1}\)到\(S_{2}\)的往返延迟,
需要对原始数据包进行封装:第一层, 以\(S_{1}\)的IP为目的IP, 第二层,
以\(S_{2}\)的IP为目的IP, 第三层, 再次将\(S_{1}\) 的IP作为目的IP.
数据包就会按照\(S_{1} => S_{2} => S_{1}\)的方式进行传递. 我们
可以通过此种方法得到粗略的往返延迟.

为什么说粗略呢, 这里,
并不是在交换机\(S_{1}\)上直接获取到了两次数据的间隔时间,
我们只能得到到达分析器的时间, 因为认为\(S_{1}\)到分析器来回路径是一致的,
因此可以 简单的使用分析器获得的两个时间相减得到粗略的往返延迟.

\textbf{定时器功能的实现}:

图\ref{fig:hash_backet}介绍了定时功能的实现, 每经过1s三个数据桶进行一次状态转移.
以0s到1s的时间段为例, 在此时间段内达到的数据包首先检查桶H0, 判断它的trace数据是否
属于H0, 如果属于, 则加入到桶H0中, 否则放置在桶H1中. 待1s结束, 到达1s到2s的时间段后,
我们将桶的状态进行转移, H0中的数据就可以作为慢路径的输入进行处理,
因为桶中的每条trace数据已经包含了整整1s的数据包.

\begin{figure}[htbp!]
  \centering
  \includegraphics[width=0.5\textwidth]{../img/hash_backet.png}
  \caption{hash桶设计}
  \label{fig:hash_backet}
\end{figure}


  根据图\ref{fig:hash_backet}中的状态转换方式, 设计如下两个线程, 分别进行快慢路径的
处理. 将定时器放在了快路径线程中, 通过定时器的触发来推动状态转换. 如果慢路径的
处理时间过久, 快路径部分将会被\textbf{while(!is\_slow\_over)\{\}}这样的自旋锁所阻塞,
反之也是如此. 目前程序中加入了对自旋锁阻塞情况的统计, 用以确定和解决程序的瓶颈.

\begin{lstlisting}[caption=数据结构,frame=tlrb]{Name}
hash_backet* H0, H1, H2;    // 三个hash数据桶
bool is_slow_over = true;   // 标示慢路径是否完成处理
Mutex lock;                 // 互斥锁, 防止快慢路径同时对数据进行修改
\end{lstlisting}

\noindent\begin{minipage}{.55\textwidth}
\begin{lstlisting}[caption=快路径线程,frame=tlrb]{Name}
clock_t start = clock();
while(!stop) {
  p = GetPacket();
  if(H0 found trace) {
    将p添加至H0中;
  } else {
    将p添加至H1中;
  }

  // 超过1s
  while(clock() - start > 1000)
  {
    // spinlock 检查慢路劲处理是否结束
    while(!is_slow_over){}
    TMP = H0;
    H0 = H2;
    H2 = H1;
    H1 = TMP;
    {
        MutexLock;
        is_slow_over = false;
    }
    // 更新clock
    start = clock();
  }
}
\end{lstlisting}
\end{minipage}\hfill
\begin{minipage}{.38\textwidth}
\begin{lstlisting}[caption=慢路径线程,frame=tlrb]{Name}
while(!stop) {
  while(is_slow_over){}
  // 处理慢路径

  {
    MutexLock;
    is_slow_over = true;
  }
}
\end{lstlisting}
\end{minipage}

\section{数据结构}

\textbf{GRE数据包}

\begin{figure}[htbp!]
  \centering
  \includegraphics[width=0.65\textwidth]{../img/format_gre.png}
  \caption{GRE数据包结构}
  \label{fig:gre_packet}
\end{figure}

对于到达分析器的数据包来说, 均是使用GRE(Generic Routing Encapsulation)
进行封装, GRE格式见图\ref{fig:gre_packet}.

\textbf{Trace数据}

这是分析器中存储结构: 传入分析器中的是一个个单独的数据包.
在分析器中, 相同路径上的数据包最终使用的trace数据结构来表示. trace数据结构中,
包括数据包内容(数据包内容在路径上都是相同的, 所以保存一份即可)这样的原始信息,
每一跳经过的交换机ID, 每一跳的到达时间, 元数据信息(是否环路、是否丢包、是否为探针).

\label{trace数据结构}
\begin{lstlisting}[caption=Trace数据结构,frame=tlrb]{Name}
typedef struct{
    struct in_addr ip_src; /**< 32bits 源IP地址   */
    struct in_addr ip_dst; /**< 32bits 目的IP地址 */
    uint16_t ip_id;        /**< 16bits 标识符     */
    uint8_t protocol;      /**< 8bits  协议字段   */
} __attribute__((packed)) IP_PKT_KEY_T;

enum {TRACE_CNT = 5};   /**< 只记录5跳信息 */

/**
 * @brief trace数据结构
 */
typedef struct{
    IP_PKT_KEY_T key;

    uint16_t pkt_size;   /**< 16bits 数据包大小  */

    /**
     * 32bits 收到第一个报文的时间戳: 如果使用秒级的计数单位, 是无法刻画出真实
     * 的数据包的时间情况, 这里的时间戳是毫秒级别的时间戳,
     * 记录的是基于当天0点偏移的毫秒数.
     *
     *  Timestamp记录收到报文的时间,
     *  如果对于某一跳交换机超过1秒还没收到其它的报文, 则视为丢包
     */
    uint32_t timestart;

    struct {
        uint16_t switch_id: 12;
        uint16_t hop_rcvd : 2;
        uint16_t hop_timeshift: 10; /**< 与timestart相减得到的偏移, 也为ms */
    } __attribute__((packed)) hop_info[TRACE_CNT - 1];

    uint16_t hp1_switch_id: 12; /**< 第一跳交换机id          */
    uint16_t hp1_rcvd: 2;       /**< 第一跳交换机收到的报文数  */
    uint16_t used: 1;           /**< 以hash表进行存储,
                                     记录hash表中的当前元素是否被占用. */

    uint16_t is_loop: 1;
    uint16_t is_drop: 1;
    uint16_t is_timeout: 1;
    uint16_t is_probe: 1;

    uint16_t reserved : 5;
} __attribute__((packed)) PKT_TRACE_T;
\end{lstlisting}


除此之外, 分析器中还有计数器, 记录每个规则下的数据包个数.

在程序中, 需要经过快路径、慢路径两个处理过程, 由快路径处理结束后,
进入慢路径.

下图\ref{fig:analyzer_process}是具体处理过程. 首先进行
对GRE数据包的解封操作, 解封后,
具有相同标识(通过五元组hash计算得出)的数据包会进入相同的线程中.

\begin{figure}
  \centering
  \includegraphics[width=0.6\textwidth]{../img/analyze_process.png}
  \caption{分析器处理逻辑}
  \label{fig:analyzer_process}
\end{figure}

上述的流程图只说明执行顺序, 以下图\ref{fig:analyzer_arch}为程序实际的处理流程,
英文关键字均为程序中确切的类名. 其中, Reader与Processer均可以自行配置个数.

\begin{figure}
  \centering
  \includegraphics[width=0.9\textwidth]{../img/analyzer_structure.png}
  \caption{分析器架构设计}
  \label{fig:analyzer_arch}
\end{figure}

\section{启动及退出流程}

\subsection{程序启动流程}
\label{sec:分析器启动过程}

  启动时, 首先初始化\texttt{Watcher}, 继而向控制器发送\texttt{INIT}信息,
请求出口交换机ID信息与当前的的统计规则. 收到控制器的回复后,
启动\texttt{Processer}, 而后启动\texttt{Reader}, 这
时的\texttt{Reader}开始获取\texttt{GRE}数据包, 并根据内层的五元组进行hash操作,
按结果存入到相应的队列中, 供\texttt{Processer}使用.

  在向控制器发请求时, 分析器必须带上自己的ID作为请求标识.
在控制器处理结束后, 返回的报文中也会携带有分析器的ID, 所有的分析器获取到广播后,
对比收到的响应报文中分析器ID与自身ID. 如果ID相同, 则说明是对自己的操作, 否则
丢弃该报文.

  另外, 对于返回报文中ID为0的情况, 这是一种广播的形式, 所有的分析器必须按照广播
报文进行操作.

\subsection{程序退出流程}

程序停止时, \texttt{Watcher}首先向控制器发送\texttt{QUIT}信号,
通知控制器, 当前的分析器准备停止了, 而后, 控制器做出回应.
每个控制器下监管许多分析器, 它有权也应当知道所有分析器的状态.

程序中多次使用了队列进行解耦, 如果队列为空, 那么所有的消费者将会被阻塞.
程序收到退出信号后, 一定是先将生产者全部停止, 待生产者全部停止后.
向队列中发送截止数据包, 向消费者表明生产者已经不再继续生产了,
可以停止了. 消费者会将当前队列中的数据包消耗完毕, 而后停止.

\section{分析器实现中遇到的问题及解决方案}

\subsection{毫秒级时间戳的使用}

我们通常所说的时间戳,
是指1970年01月01日00时00分00秒(北京时间1970年01月01日08时00分00秒)起至现在的总秒数.
是以秒为计数单位的.

在我们的trace数据结构中, 也采用了32位整数作为时间戳, 但是针对网络中的数据包. 它们的
传播速度是在微妙级别, 用秒做单位根本无法进行时间估计.

\subsubsection{可能的方案}

考虑过之后, 有以下两种方案:

\begin{itemize}
\item
  再来32位, 表示毫秒;
\item
  将原始的方案中时间戳的含义更改,
  使其表示从一个特定时间开始经过的总毫秒。
\end{itemize}

第一种方案, 需要修改原先的数据结构了, 向其中增加4个byte,
虽然我们目前的程序不是针对专用的硬件平台, 但是两者的数据结构定义是一致的,
具体的数据结构在\ref{trace数据结构}已经列出,
专用硬件平台中缓存大小为128byte, 可以存储4个这样的trace数据,
根据我们先前的定义,
假如每个对象增加了4个字节, 原始的128byte的CPU缓存就不够存放4个trace了,
会造成性能下降.

第二种方案, 从特定时间开始, 记录基于这个时间的偏移, 那么就会产生一个问题:
32位(大约有20亿)的时间戳以毫秒为单位, 到底能够表示长时间.

这里计算了32位的数据能表示的天数.

$$ \frac{2 \times 10^{9} ms }{24 \times 3600 \times 1000} = 23 Days$$

粗略的计算表明, 如果使用毫秒表示时间戳, 最多也就只能表示23天,
如果使用\texttt{unsigned\ int}, 时间翻倍也就是46天,
如果使用每一年的年初作为起始, 偏移最多能表示46天, 完全达不到我们的要求.

\subsubsection{最终方案}

使用32位时间戳表示当天偏移的毫秒数, 再加一个字段保存日期.
但这个字段不放在分析器的trace结构中, 而是放置在数据库里.

采用这个方案的原因有以下两点:

\begin{enumerate}
\def\labelenumi{\arabic{enumi}.}
\item
  32位的时间戳已经足够表示1天内偏移的毫秒数;
\item
  整个系统的实时性很强, 误差也不会超过10s, 存入数据库时,
  将数据库当天的日期作为 日志日期写入到字段中, 这样,
  可以保证我们在搜索时获得正确的时间戳, 也不需要考虑空间占用问题, 并且,
  这个新增的日期字段可以作为索引来使用.
\end{enumerate}

日期字段可以通过MySQL的trigger进行设置.


\begin{lstlisting} [caption=Triger设置]{SQL}
CREATE TRIGGER `fdate_set` BEFORE INSERT ON `tbl_trace_data`
FOR EACH ROW BEGIN
  SET NEW.fdate = CAST( DATE_FORMAT(NOW(),'%Y%m%d') AS UNSIGNED);
END
\end{lstlisting}

\subsubsection{细节问题}

这样存储会有一个问题:如果出现了快到第二天0点时存入了数据,
而设置日期时已经到了第二天, 例如:2018-03-22 23:59:55 时准备存入数据库,
之后数据库中的fdate被设置为了2018-03-23, 这是一种不匹配.

这个问题也有解决方式:可以通过判断时间戳的偏移, 如果偏移过大, 则认为它是
第一天的数据, 也就是03-22的数据, 上面只是列了最简单的Trigger设置.

\subsection{热重启与容灾问题}
\label{sec:热重启与容灾问题}

在考虑到控制器与分析器的交互后, 直接启动分析器的情况已经在
分析器启动过程\ref{sec:分析器启动过程}中提到这个问题,
但是作为线上项目, 热重启与容灾也成为了一个问题.

目前, 我的打算是将分析器部署在五台服务器上, 线上有三台分析器, 另有两台分析器进行容灾.

\subsubsection{热重启}

重启过程是一个分析短暂暂停的过程, 即使我们想要使其继续接受流量,
我们也要做好分析速率变慢的准备.


进行重启操作时, 采用分批重启的形式. 假如我们需要热重启分析器1,
首先配置交换机, 将网络流量导入到分析器2,3中, 重启分析器1完成后,
再将流量导入到分析器1,3中, 依次类推. 如果容灾机器空闲,
可以考虑导入先导入到容灾机器中.

\subsubsection{容灾}

当分析器遇到了故障, 需要检测时, 首先启动容灾服务器, 继而修改交换机配置,
将故障分析器的流量全部切换到容灾分析器后, 停机检测.

\subsection{定时器与signal}

在当前的程序中, 定时上传功能可能只需要10s, 或是20s才上传一次,
单独开线程进行上传操作显得十分浪费.

最终, 我决定使用Linux中的定时器功能. 配置程序, 使得Linux
定时向程序发送\texttt{SIGALRM}信号,
这样, 我们在信号处理函数中添加上传操作即可.

不过, 这样做会造成一个问题, POSIX的信号处理函数只接受一个参数,
我们需要使用全局变量才可以在信号函数中访问程序中的数据.

\subsection{与控制器的交互部分}

与控制器的交互分为两个部分, 发送和接受. 这两个部分均在watcher中实现.

\textbf{发送}:向消息队列中发送信息, 该消息队列由向控制器监听.

使用消息队列主要是因为可能有多个分析器需要上传请求, 队列是一个比较简单的方式.

\textbf{接受}:从某个频道中读取控制器命令, 如果是属于自己的命令信息, 则进行相应操作.

如果控制器采取一对多的连接方式, 其实是没多少必要的, 与分析器交互主要是增加删除规则,
增加删除出口交换机ID, 而这样的操作基本是对所有分析器进行的. 使用
订阅-发布模式可以简化控制器的逻辑, 提高效率, 并且有利于分析器数量上的扩展.

与控制器的交互部分全部采用\texttt{Redis}解耦,
其中使用了\texttt{Redis}的\texttt{消息队列}和\texttt{订阅发布}两种接口.
两种接口的使用与调试方式如下:

\textbf{消息队列}: 由 分析器 =\textgreater{} 控制器.

\begin{lstlisting} [caption=分析器推送]{Name}
> 127.0.0.1:6379> LPUSH foo 1234 # 分析器PUSH
> (integer) 1

> 127.0.0.1:6379> LPOP foo  # 控制器接受
> "1234"
\end{lstlisting}


\textbf{订阅发布}: 由控制器发送广播, 许多分析器监听相同的频道.


\begin{lstlisting} [caption=控制器下发]{Name}
> 127.0.0.1:6379> SUBSCRIBE foo        # 首先启动监听foo频道
> Reading messages... (press Ctrl-C to quit)
> 1) "subscribe"
> 2) "foo"
> 3) (integer) 1
> 1) "message"
> 2) "foo"
> 3) "12334"

> 127.0.0.1:6379> PUBLISH foo 12334   # 控制器向foo频道中发送
> (integer) 1
\end{lstlisting}



% }}} 分析器实现方案

% {{{ 控制器的设计与实现
\chapter{控制器的设计与实现}

\section{控制器与分析器的接口设计}
\label{sec:分析器, 控制器接口设计}

\subsection{启动过程}

每台分析器启动时, 首先向控制器发送请求. 获取计数器规则, 以及出口交换机ID,
这里是向控制器监听的队列中发送初始化请求.

\begin{lstlisting}[caption=分析器初始化请求]{Name}
{
  "ACTION": "INIT",
  "ANALYZER_ID": 124         // 分析器ID, 由分析器的配置文件进行配置
}
\end{lstlisting}

控制器收到报文后, 针对当前的分析器ID, 进行返回, 同时带上出口交换机的ID,
以及计数器信息, 通过控制器进行配置, 可以简化分析器部分.

\begin{lstlisting}[caption=控制器返回报文]{Name}
{
  "ANALYZER_ID": 124,           // ID为0, 表示这是一种广播操作.
  "MESSAGE": {                  // 报文主体
    "COMMOND": "INIT",          // 初始化
    "SWH_ID": [
        14, 23, 19, 40
    ],
    "COUNTER": [                // 计数器的filter
        {
            "CNT_ID" : 10,
            "SRC_IP": "192.118.0.2",
            "DST_IP": "192.119.0.1",
            ...
        }
    ]
  }
}
\end{lstlisting}

\subsection{更新出口交换机与计数器信息}

由控制器主动进行发送指令, 触发分析器的重载过程, 重载会导致分析速率下降,
在热重启与容灾部分\ref{sec:热重启与容灾问题}提供了一种解决方案.

\begin{lstlisting}[caption=控制器下发指令]{Name}
{
  "ANALYZER_ID": 13,        // ID为0, 表示这是一种广播操作.
  "MESSAGE": {              // 报文主体
    "COMMOND": "RELOAD",    // 重载过程, 这里认为是热重启
    "SWH_ID": [
        14, 23, 19, 40
    ],
    "COUNTER": [            // 计数器的filter
        {
            "CNT_ID" : 1
            "SRC_IP": "192.118.0.2",
            "DST_IP": "192.119.0.1"
        }
    ]
  }
}
\end{lstlisting}

增加删除Counter规则, 以及出口交换机ID, 这一操作应该是针对每个分析器进行, 以下为
具体的命令. 表\ref{tbl:message}列出了所有可用的接口形式.


\begin{table}[]
    \centering
    \caption{控制器命令}
    \label{tbl:message}
    \begin{tabular}{lllll} \hline
    COMMAND      & 解释         & 具体报文                         \\ \hline
    ADD\_RULE    & 增加规则     & "COUNTER": {[}\{"ID":1\} .. {]}  \\
    DEL\_RULE    & 删除规则     & "COUNTER": {[}1, 2,3{]}          \\
    ADD\_SWH\_ID & 增加交换机ID & "SWH\_ID" : {[}13, 4, 4{]}       \\
    DEL\_SWH\_ID & 删除交换机ID & "SWH\_ID" : {[}23,{]}            \\ \hline
    \end{tabular}
\end{table}

\begin{lstlisting}[caption=控制器增加删除规则]{Name}
{
  "ANALYZER_ID": 0,         // ID为0, 表示这是一种广播操作.
  "MESSAGE": {              // 报文主体
    "COMMOND": "ADD_RULE",  // 增加删除规则
    "SWH_ID": [
        14, 23, 19, 40
    ],
    "COUNTER": [           // 计数器的filter
        {
            "CNT_ID" : 1
            "SRC_IP": "192.118.0.2",
            "DST_IP": "192.119.0.1"
        }
    ]
  }
}
\end{lstlisting}

\section{控制器与展示页面接口设计}

\subsection{添加删除规则}

\url{https://127.0.0.1:9999/v1/rules}

请求参数

\begin{table}[]
    \centering
    \caption{添加删除规则参数}
    \label{tbl:add_del_rule}
    \begin{tabular}{lll} \hline
    Field              & Type   & Description                 \\ \hline
    act                & String & 只能使用 'ADD|DEL'              \\
    rule\_id           & Number & 在DEL操作时必须填写                 \\
    rule\_name         & String & 在ADD操作时必须填写                 \\
    ip\_srcoptional    & String & Default value: 0.0.0.0      \\
    ip\_dstoptional    & String & Default value: 0.0.0.0      \\
    protocoloptional   & Number & 报文中的协议类型Default value: -1   \\
    switch\_idoptional & Number & 镜像报文的交换机Default value: -1 \\ \hline
    \end{tabular}
\end{table}

\subsection{请求Trace数据信息}

\url{https://127.0.0.1:9999/v1/tracce_filter}

\begin{table}[]
    \centering
    \caption{请求Trace数据}
    \label{tbl:get_trace}
    \begin{tabular}{lll} \hline
    Field             & Type   & Description            \\ \hline
    start\_time       & String &                        \\
    end\_time         & String &                        \\
    ip\_srcoptional   & String & Default value: 0.0.0.0 \\
    ip\_dstoptional   & String & Default value: 0.0.0.0 \\
    protocoloptional  & Number & Default value: -1      \\
    is\_loopoptional  & Number & Default value: -1      \\
    is\_dropoptional  & Number & Default value: -1      \\
    is\_probeoptional & Number & Default value: -1      \\ \hline
    \end{tabular}
\end{table}

\subsection{请求counter数据信息}

\url{https://127.0.0.1:9999/v1/counter_filter}

\begin{table}[]
    \centering
    \caption{请求Counter数据}
    \label{tbl:get_counter}
    \begin{tabular}{lll}  \hline
    Field       & Type   & Description      \\ \hline
    start\_time & String &                  \\
    end\_time   & String &                  \\
    rule\_id    & Number & Default value: 0 \\ \hline
    \end{tabular}
\end{table}

\subsection{请求所有规则}

\url{https://127.0.0.1:9999/v1/rules}

这是一个\texttt{GET}请求, 查询所有的规则, 所有的rule在访问页面初始化时就应该被获取到,
而后通过\texttt{rule\_id}查找counter数据.


\section{控制器的内部实现}

控制器采用\texttt{Python}实现. 具体使用了\texttt{Tornado}框架.
它向下控制分析器, 向上为应用程序提供接口.

\begin{figure}[htbp!]
  \centering
  \includegraphics[width=0.9\textwidth]{../img/controller_arch.png}
  \caption{控制器构造}
  \label{fig:contoler_arch}
\end{figure}

控制器中, 涉及到三个方面:

\textbf{与存储器交互}:我们的存储器使用\texttt{MySQL},
Python程序可以方便的进行读写, 允许用户读取Trace记录, 读取统计信息.
允许用户读写统计规则, 以及构造探针

\textbf{与分析器交互}:通过\texttt{Redis}\cite{redis}进行解耦, 每个分析器启动时,
需要首先向控制器发送请求, 获得控制器的允许后, 才能启动.
如果用户更新了统计规则, 控制器也会立刻下发到所有的的分析器中.
在分析器准备结束时, 也同样需要告知控制器. 未来的控制器可能还需要添加
对于每个分析器的管理功能.

\textbf{与用户进行交互}: 通过访问页面发送请求, 进行交互. 采用基于HTTP协议的API,
程序功能也可以轻松的进行扩展.


\section{有关程序性能的问题}

  对于\texttt{CPython}来说, 由于全局中断锁也就是\texttt{GIL}的存在, \texttt{CPython}
无法充分的利用多核. 因此, 编写控制器时, 我选择单线程, 也因此选择了\texttt{Tornado}.

 不过所有用户均是在一个线程中进行轮转, 如果某个用户请求时间过久,
将会严重影响其他用户的查询体验.
% }}} 控制器的设计与实现

% {{{ 存储器与访问页面的设计与实现
\chapter{存储器与访问页面的设计与实现}

\section{存储器的设计与实现}

\subsection{数据库设计}
\label{sec:数据库设计}

\subsubsection{trace数据存储}

存储trace数据的库表分为三大部分:

\begin{itemize}
    \setlength\itemsep{0.1em}
    \item 第一部分是数据包原始信息, 包括数据包头部以及负载信息;
    \item 第二部分是每一跳的信息, 比如时间戳、源MAC地址. 每个trace数据跳数不同,
            所以保存时, 将所有跳的信息结合在一起存储;
    \item 第三部分是元数据信息, 表示这个trace信息是否有环、是否丢包、
            是否为探针数据.
\end{itemize}


\begin{table}[]
    \centering
    \caption{tbl\_traffic\_data}
    \begin{tabular}{llll}   \hline
    Field          & Type          & Comment                   \\ \hline
    id             & int(11)       &                           \\
    s\_ip          & int(11)       & 源IP                       \\
    d\_ip          & int(11)       & 目的IP                      \\
    protocal       & int(11)       & 协议类型                      \\
    generate\_time & timestamp     & 产生时间                      \\
    trace\_data    & varchar(1024) & trace数据信息, 保存为二进制字符串    \\
    fdate          & int(11)       & 存入日期, 如果数据量过大则使用索引 \\
    is\_loop       & int(11)       & 是否有环                      \\
    is\_drop       & int(11)       & 是否丢包                      \\
    is\_probe      & int(11)       & 是否为探针                    \\ \hline
    \end{tabular}
    \label{tbl_traffic_data}
\end{table}



\subsection{Counter计数器表格的设计}

存储计数器的数据表相对简单, 主要记录分析器ID, 计数值以及产生时间.

\begin{table}[]
    \centering
    \caption{tbl\_counter}
    \label{tbl_counter}
    \begin{tabular}{lll} \hline
    Field          & Type          & Comment \\ \hline
    id             & int(11)       &         \\
    counter\_name  & varchar(1024) & 计数器名称   \\
    generate\_time & timestamp     & 数据产生时间  \\
    analyer\_id    & int(11)       & 分析器ID   \\
    cnt            & int(11)       & 计数值     \\
    fdate          & int(11)       & 数据产生日期  \\ \hline
    \end{tabular}
\end{table}

目前的分析器中, 想要定时10s进行发送计数器信息, 所以一天的数据大约有
$$\frac{24 \times 3600 s}{10s} = 8640 Rows $$

如果我们有3台交换机的话, 10个计数规则的话, 每天将会有250,000条记录,
每个月大约有7,500,000条.

MySQL主要支持百万级别的数据, 因此, 在正式投入使用时,
整个程序需要使用分区表(按月分区).
如果有可能我们最好将粒度设置为20s, 可以减少一半的存储占用.

我建议数据保存时间最好不要超过1个月, 有了问题需要尽快解决.


\subsection{存储trace数据中路径信息的考量}

由于对于某个数据包的整条路径, 我们是需要进行存储的. 但是,
路径存储到数据库后, 我们不打算提供对于数据路径的检索功能. 这样,
可以将所有字段结合在一起放置在数据库中.

对于该字段, 曾经我考虑使用\texttt{Protobuf}与\texttt{JSON}两种.
最终选择了\texttt{JSON}, 主要有以下两个原因:

\begin{enumerate}
\def\labelenumi{\arabic{enumi}.}
\item
  \texttt{JSON}不只是在存储中用到了,
  在\texttt{Watcher}与控制器交互过程中, 也全部使用了\texttt{JSON}. 因此,
  使用\texttt{JSON}能减少项目依赖的类库数量;
\item
  \texttt{Protobuf}在存入数据库时, 需要采用特定的数据结构. 存入取出时,
  开发者并不容易使用肉眼调试分析.
\end{enumerate}

鉴于此, 我放弃了内存占用量较小的\texttt{Protobuf},
转而使用易于读取理解的\texttt{JSON}.


\section{访问页面设计与实现}

\subsection{访问页面设计}

下图\ref{fig:traffic_data_show}, \ref{fig:counter_data_show}
分别展示了数据流量的显示以及计数器统计结果的显示,

\begin{figure}[htbp!]
  \centering
  \includegraphics[width=0.9\textwidth]{../img/traffic_data_show.png}
  \caption{路径信息}
  \label{fig:traffic_data_show}
\end{figure}

\begin{figure}[htbp!]
  \centering
  \includegraphics[width=0.9\textwidth]{../img/counter_data_show.png}
  \caption{计数器数据}
  \label{fig:counter_data_show}
\end{figure}

\subsection{有关预警方式选择}

  预警也是本程序应有的功能, 为了开发方便, 以及未来其他接入功能的使用,
我采用了\texttt{PushBear}服务, 用户可以直接扫描二维码关注预警频道.
而后将会收到预警信息. 用户可以在任意时刻取消订阅.

\begin{figure}[htbp]

\centering
\begin{minipage}{0.76\textwidth}

    \includegraphics[width=1\textwidth]{../img/pushbear.png}
    \caption{PushBear官网}
    \label{fig:pushbear}

\end{minipage} \hfill
\begin{minipage}{0.22\textwidth}

\centering
    \includegraphics[width=1\textwidth]{../img/push_example.jpg}
    \caption{推送样例}
    \label{fig:push_example}
\end{minipage}

\end{figure}

如图\ref{fig:pushbear}为官网, 右侧\ref{fig:push_example}为推送样例.

%  https://pushbear.ftqq.com/admin/#/
% }}} 存储器与访问页面的设计与实现

% }}} 实现方案

%  {{{ 程序部署与性能评测
\chapter{程序部署与性能评测}
\section{程序部署}

  本次的程序设计主要分为三个部分, 分析器、存储器以及控制器, 展示页面由控制器
负责加载. 承接上文分析器控制接口设计\ref{sec:分析器, 控制器接口设计}, 分析器、
控制器、存储器之间是不存在耦合的, 均可进行单独的部署.

  它们之间唯一需要的只有TCP连接. 另外, 有关Redis服务器, 我选择将其与存储器放置在一台
服务器上. 最主要的原因就是Redis以及MySQL的存在, 就是为了连接分析器与控制器,
放置在一台服务器上符合它们的作用.

  分析器可以在多台PC进行部署, 这是由于一条路径上的所有数据包都会根据hash操作
分配至同一个分析器中. 可以利用这一特性, 进行横向扩展来提高系统的吞吐量.

  有关详细的部署操作, 在代码的文档\texttt{orig\_md/测试环境搭建}\cite{Niftyflow}
中有详细记录, 在此不进行赘述.

\section{评测要点与评测环境}

  程序中, 实现的功能有下面这些:

\begin{itemize}
    \item 对分析器:包括有网卡数据读取、快慢路径处理数据、有问题的trace数据入库、
计数器报告、报警信息发送;
    \item 对存储器:读写有问题的trace数据、读写计数值;
    \item 对控制器:分析器的初始化、提供数据访问接口、添加删除规则、探针的发送;
\end{itemize}

  对于存储器来说, 只负责存储异常的数据包信息以及提供访问能力, 建立合适的索引之后,
数据库可以保持较好的性能. 控制器只负责控制分析器与提供接口, 数据量并不大.
三者中对性能要求较高的部分为分析器, 尤其是数据包的读取与处理能力, 决定着整个
程序的吞吐量, 测试的主要工作也只针对分析器.


测试环境的服务器配置如表\ref{tbl:server_config}所示.

\begin{table}[]
    \centering
    \caption{服务器配置}
    \label{tbl:server_config}
    \begin{tabular}{ll} \hline
    Field    & Type          \\ \hline
    Linux发行版 & CentOS 7.4 \\
    内核版本   &   3.10.0 \\
    CPU:     & Intel(R) Xeon(R) CPU E5-2690 v4 @ 2.60GHz(28核) \\
    内存:    & 32G                                       \\
    数据来源 & 使用pcap监听lo 采用tcpreplay指定测试数据以及发送速率. \\ \hline
    \end{tabular}
\end{table}


\section{评测过程}

\begin{itemize}
    \item 首先启动分析器程序(分析器由1个Reader, 8个Processor组成);
    \item tcpreplay进行大量的数据包发送, 评测时的输入3,000,000pps(packets per second);
    \item 评测大概进行了5分钟, 而后我将程序关闭并退出.
\end{itemize}

\section{评测结果及分析}

 下面两张图对应的均为日志记录为\texttt{INFO}的情况. 对于日志记录为\texttt{DEBUG}
的情况, 生成的日志过于庞大, IO占了很大一部分性能, 要比INFO时的性能损失一些.

\begin{figure}[htbp!]
  \centering
  \includegraphics[width=0.8\textwidth]{../img/队列情况-line.png}
  \caption{队列中数据变化}
  \label{fig:queue_pic}
\end{figure}

\begin{figure}[htbp!]
  \centering
  \includegraphics[width=0.8\textwidth]{../img/每个线程处理情况-new.png}
  \caption{各个线程处理情况}
  \label{fig:process_pic}
\end{figure}


通过观察\ref{fig:queue_pic}, 可以看得出, 使用lo进行数据接收时, 队列中的数据量始终
保持在0-2之间. 这也就说明了一点, 当前程序的瓶颈是在\texttt{pcap\_open\_live}阶段.
这里我只测试了\texttt{pcap}库, 未来如果改用\texttt{dpdk}或许能达到更好的效果.

通过观察\ref{fig:process_pic}, 所有process线程的处理速度基本是稳定的,
处理速度上的不同, 原因只是hash函数选择的问题, 某些线程被分到了比较多的数据包.
也可以看的出, 当前的处理速率大概是88万pps, 也就是平均每个线程有10万pps.
这里并不是说单个线程的处理只有10万pps, 只能说明, 我们每秒到达的数据包只有这么多.

具体的评测过程以及日志分析脚本请查看
\texttt{analyzer/README.md}\cite{Niftyflow}中的性能评测部分,我介绍了相对详细
的评测过程以及结果分析.
% }}} 程序部署与性能评测

\chapter{总结与展望}

\section{论文总结}

  本次的设计, 实现了一个针对大型DCN的可扩展数据包级检测系统, 基于商用交换机的
"匹配镜像"机制, 可以获取捕获DCN中的数据包, 以及发送探针重现数据包的路径.
项目也有详细的部署文档, 可以快速部署到DCN中, 同时, 也提供了查询和配置接口.

  在微软提出Everflow设计时, 并没有开放源代码, 当前的市面上也没有针对Everflow的
开源实现. 本次的设计也填补了这项空白, 整个项目作为开源工程, 接受大家的检验并
提供支持.

  实验室中基于专有硬件平台的分析工具正在实现中. 此次我的毕业设计, 实现了最初提出的
功能后, 对程序性能也进行了评估, 为实验室的工作提供了一份参照.

\section{工作展望}

  目前的工作中, 是在数据包读入部分存在瓶颈, 未来, 可能会使用\texttt{dpdk}
以及\texttt{RSS}(Receive side scaling)\cite{rss}机制, 允许一个分析器使用多个CPU
核接受数据包.

  目前的部署只在单台DCN中进行了部署, 程序的鲁棒性以及数据处理能力还有待进一步
测试. 程序中的内存泄露问题已经解决(使用Valgrind\cite{Valgrind}),
但从我的调研结果来看, 针对\texttt{DPDK}程序进行内存泄露检测时,
\texttt{Valgrind}需要打补丁进行编译, 所以\texttt{DPDK}程序的内存安全性不能保证.
这些工作待有了真实环境之后, 可以着手开始.

  当前对于错误信息, 程序需要通知运维人员进行处理, 未来, 也会为程序增加自动化的
错误探测功能, 主动解决网络问题并生成报告.
